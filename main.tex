\documentclass[t, 10pt, handout, aspectratio=169]{beamer}
\usepackage{lipsum}
\usepackage{tikz}
\usepackage{pgfplots, pgfplotstable}
\usepackage{booktabs}
\usepackage{amsmath,bm}
\usepackage{mdframed}


\pgfplotsset{compat=1.14}

%\usetheme[framenumber,totalframenumber]{QU}
\usetheme[color=blue,framenumber,totalframenumber, footline, footertext]{KU}


\title[Introduction to Tensor]{Introduction to \texttt{Tensor}}
\subtitle{Intelligent Computing for Computational Intelligence}
\author[yanglet]{Xiao-Yang Liu}
\institute[CU]{Columbia University}

\footertext{\url{www.tensorlet.com}}


\date[\number\month/\number\day/\number\year]{\today}


\begin{document}

\begin{frame}
  \titlepage
\end{frame}

\begin{frame}{Agenda}
\begin{itemize}
    \large \item \textcolor{red}{Background}
    \large \item {Tensor Decompositions (CP, Tucker, and Tensor-Train)}
    \large \item{Transform-based Tensor Model and Applications}
    \large \item{Tensor Computations}
\end{itemize}
\end{frame}

\begin{frame}{Background}
\large Multidimensional data of exceedingly huge volume, variety and structural richness become ubiquitous across disciplines in engineering and data science:
\begin{itemize}
    \item multimedia data like speech and video
    \item remote sensing data
    \item medical and biological data
    \item seismic data
\end{itemize}
\vskip 0.2\margin
\large Some data can have more meaningful representation using multi-way arrays -- \textbf{tensor}, rather than matrices (two-way arrays).
\end{frame}

\begin{frame}{What is tensor?}
\vskip -1ex
\begin{figure}
	\centering  
	\includegraphics[height=0.7\paperheight]{figs/tensor_shape}
	\label{fig:tensor_shape}
\end{figure}
\end{frame}

\begin{frame}{What is tensor?}
\vskip -1ex
\begin{figure}
	\centering  
	\includegraphics[height=0.7\paperheight]{figs/tensor_sample}
	\label{fig:tensor_sample}
\end{figure}
\end{frame}

\begin{frame}{Tensor fibers}
\vskip -3ex
\begin{figure}
	\centering  
	\includegraphics[width=\linewidth]{figs/tensor_fibers}
	\label{fig:tensor_fibers}
\end{figure}
\end{frame}

\begin{frame}{Tensor slices}
\begin{figure}
	\centering  
	\includegraphics[width=\linewidth]{figs/tensor_slices}
	\label{fig:tensor_slices}
\end{figure}
\end{frame}

\begin{frame}{Tensor unfolding}
\vskip -3ex
\begin{columns}[c]  %开始进入分栏环境,居中设置
\column{0.4\linewidth}  
\begin{figure}
	\centering  
	\includegraphics[height=0.75\paperheight]{figs/tensor_unfolding}
	\label{fig:tensor_unfolding}
\end{figure}

\column{0.6\linewidth}  
\begin{figure}
	\centering  
	\includegraphics[width=\linewidth]{figs/tensor_unfolding2}
	\label{fig:tensor_unfolding2}
\end{figure}
\centering
$\mathbf{A}_{(i)}$ means mode-$i$ unfolding.
\end{columns}  %分栏环境结束

\end{frame}



\begin{frame}{Agenda}
\begin{itemize}
    \large \item {Background}
    \large \item \textcolor{red}{Tensor Decompositions (CP, Tucker, and Tensor-Train)}
    \large \item{Transform-based Tensor Model and Applications}
    \large \item{Tensor Computations}
\end{itemize}
\end{frame}

\begin{frame}{CP Decomposition}
\vskip -1ex
\begin{figure}[t]
	\centering  
	\includegraphics[width=\linewidth]{figs/cp_arch}
	\label{fig:cp_arch}
\end{figure}
\begin{columns}
\column{0.5\linewidth}
\vskip -10ex
\begin{align*}
\underline{\mathbf{X}}&\approx\sum_{r=1}^{R}\lambda_{r}\mathbf{b}_{r}^{(1)}\circ\mathbf{b}_{r}^{(2)}\circ\cdots\circ\mathbf{b}_{r}^{(N)} \\
&=\underline{\mathbf{\Lambda}}\times_{1}\mathbf{B}^{(1)}\times_{2}\mathbf{B}^{(2)}\cdots\times_{N}\mathbf{B}^{(N)} \\
&=[\![\mathbf{\Lambda};\mathbf{B}^{(1)},\mathbf{B}^{(1)},\cdots,\mathbf{B}^{(N)}]\!]
\end{align*}

\column{0.5\linewidth}
\vskip -8ex
\begin{align*}
\mathbf{X}_{(1)}&=\mathbf{A}\mathbf{\Lambda}(\mathbf{C}\odot\mathbf{B})^{T}+\mathbf{E}_{(1)} \\
\mathbf{X}_{(2)}&=\mathbf{B}\mathbf{\Lambda}(\mathbf{C}\odot\mathbf{A})^{T}+\mathbf{E}_{(2)} \\
\mathbf{X}_{(3)}&=\mathbf{C}\mathbf{\Lambda}(\mathbf{B}\odot\mathbf{A})^{T}+\mathbf{E}_{(3)}
\end{align*}

\end{columns}
\end{frame}

\begin{frame}{Tucker Decomposition}
\vskip -2ex
\begin{figure}[t]
	\centering  
	\includegraphics[width=0.95\linewidth]{figs/tucker_arch}
	\label{fig:tucker_arch}
\end{figure}
\vskip -2.5ex
\begin{mdframed}[backgroundcolor=brown!20,roundcorner=8pt,leftmargin=60pt,rightmargin=60pt]
\centering
$
\underline{\mathbf{Y}}=\underline{\mathbf{G}}\times_{1}\mathbf{A}\times_{2}\mathbf{B}\times_{3}\mathbf{C}+\mathbf{E}=[\![\mathbf{G};\mathbf{A},\mathbf{B},\mathbf{C}]\!]+\underline{\mathbf{E}}
$
\end{mdframed}
\vskip -0.5ex
\begin{align*}
\mathbf{X}_{(1)}&\approx\mathbf{A}\mathbf{G}_{(1)}(\mathbf{C}\otimes\mathbf{B})^{T}\\
\mathbf{X}_{(2)}&\approx\mathbf{B}\mathbf{G}_{(2)}(\mathbf{C}\otimes\mathbf{A})^{T}\\
\mathbf{X}_{(3)}&\approx\mathbf{C}\mathbf{G}_{(3)}(\mathbf{B}\otimes\mathbf{A})^{T}
\end{align*}
\end{frame}

\end{document}
